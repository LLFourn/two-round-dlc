\documentclass[runningheads]{llncs}

	% \addtolength{\oddsidemargin}{-.875in}
	% \addtolength{\evensidemargin}{-.875in}
	% \addtolength{\textwidth}{1.75in}

	% \addtolength{\topmargin}{-.875in}
	% \addtolength{\textheight}{1.75in}

\usepackage[utf8]{inputenc}
\usepackage[n,advantage,operators,sets,adversary,landau,probability,notions,logic,ff, mm, primitives,events,complexity,asymptotics,keys]{cryptocode}
\usepackage{wrapfig}
% \usepackage[english]{babel}
% \usepackage{amsthm}



% \theoremstyle{definition}
% \newtheorem{claim}{Claim}[section]
% \newtheorem{conjecture}{Conjecture}[section]
% \newtheorem{definition}{Definition}[section]
% \newtheorem{theorem}{Theorem}[section]
% \newtheorem{lemma}{Lemma}[section]
% \newtheorem{remark}{Remark}




\begin{document}

\title{How to Make a Prediction Market on Twitter with Bitcoin}
\author{Lloyd Fournier}
\institute{\email{lloyd.fourn@gmail.com}}
\date{September 2019}

\maketitle

\begin{abstract}

  % Prediction markets

  % Prediction markets have often been cited as a compelling use of distributed ledgers.

In his work \emph{Discreet Log Contracts} (DLC) Dryja presents a compelling design for smart contracts that settle based on real world events. We show how to optimise the construction for events with binary outcomes. Our protocol requires only two rounds of communication for the contract setup and only two on-chain transactions. The setup messages are small enough so that they can be encoded in a single tweet or text message. This circumvents the need for a prediction market ``exchange'' and for a peer-to-peer communication channel to execute the protocol. Instead, users use their existing social media and messaging channels to both find other users to bet with and to exchange the necessary cryptographic information.
\end{abstract}

\section{Introduction}

In the blockchain world, an ``Oracle'' is a trusted party who attests to the state of the real world so that \emph{smart contracts} may settle based on real world events. There are a variety of proposals for how oracles can perform this function. The most compelling model on Bitcoin-like systems are described the Dryja's \textit{Discreet Log Contracts} (DLC)\cite{DLC}. In this model, oracles don't interact with the blockchain nor do they even necessarily interact with the parties involved in the contract. Instead, oracles attest to an outcome by publishing secret information they had previously committed to revealing should that outcome occur. Specifically, in DLCs the oracle commits to revealing a Schnorr signature's $s$ value on a fixed nonce with distinct message for each possible outcome. This amounts to committing to a set of group elements each corresponding to an event outcome and then later revealing the discrete logarithm of the corresponding element to attest to the particular outcome.

The most attractive feature off DLCs in contrast to other proposals is that the oracle is an \emph{oblivious trusted third party}\cite{cryptoeprint:2011:319}. Despite being trusted, they are kept oblivious to any concrete protocol instance consuming their attestations. Importantly, this holds true even if the oracle is actively monitoring the blockchain as nothing the oracle commits to or reveals is directly used in the smart contracts published on the blockchain.

\subsection{Our Contribution}

We present an optimisation of the original construction for binary outcome events which makes it practical to execute the protocol through text based communication channels like social media and messaging applications. Our protocol has only two rounds of communication and messages small enough to fit into a single 240 character tweet (assuming a maximum of 380 bytes per tweet \cite{base2048}). Despite the messages of the protocol being published publicly, we manage to preserve privacy properties of the protocol by using a Diffie-Hellman key exchange to encrypt and randomise the keys that appear in the transaction outputs.

\section{The Protocol}
\newcommand{\Rec}{\textsf{Rec}}
\newcommand{\bet}{\beta}
\newcommand{\hatsigma}{\hat{\sigma}}
\newcommand{\Fund}{\textsf{Fund}}
\newcommand{\Outcome}{\textsf{Outcome}}
\newcommand{\KeyGen}{\textsf{KeyGen}}
\newcommand{\win}{\textsf{win}}
\newcommand{\Verify}{\textsf{Vrfy}}
\newcommand{\Tx}{\textsf{Tx}}
\newcommand{\EncVer}{\textsf{EncVrfy}}
\newcommand{\Pdleq}{\pcalgostyle{P}_{\textsf{DLEQ}}}
\newcommand{\Vdleq}{\pcalgostyle{V}_{\textsf{DLEQ}}}
\newcommand{\change}{\textsf{change}}
\newcommand{\val}{\textsf{val}}
\newcommand{\OPCHECKMULTISIG}{\texttt{CHECKMULTISIG}}
\newcommand{\OPCMS}{\texttt{OP\_CMS}_{\text{1-of-2}}}
\newcommand{\fee}{\textsf{fee}}
\newcommand{\Sign}{\textsf{Sign}}
\newcommand{\EncSign}{\textsf{EncSign}}
\newcommand{\Rx}{R_\texttt{x}}
\newcommand{\DecSig}{\textsf{DecSig}}
\newcommand{\TxGen}{\textsf{TxGen}}
\newcommand{\eventid}{\textsf{event\_id}}
\newcommand{\PRG}{\textsf{PRG}}
\newcommand{\HKDF}{H_{\textsf{KDF}}}
\newcommand{\G}{\mathbb{G}}
\newcommand{\Enc}{\textsf{Enc}}
\newcommand{\Dec}{\textsf{Dec}}
\newcommand{\VrfyWitness}{\textsf{VerifyWitness}}
\newcommand{\GenFund}{\textsf{GenFund}}
\newcommand{\GenWitness}{\textsf{Witness}}


The protocol begins with Alice posting a message on a public message board like Twitter containing a public key, what she wants to bet on, how much she wants to bet and which inputs she will to fund the bet. Users willing to offer Alice a bet on this event then post encrypted replies. The offers contain a public key and payload encrypted by the Diffie-Hellman shared secret. Inside the payload are signatures on inputs for the $\Fund$ transaction. The output of the $\Fund$ transaction associated with the bet uses the pseudorandomly blinded public keys of the two parties combined with the oracle's public keys corresponding to the outcome of each event. Either party learning the discrete logarithm of the event public key (denoted as $O_0$ or $O_1$) will be able to claim the coins.

We denote Bitcoin's Secp256k1 elliptic curve group as $\G$ its order as $q$ and its fixed generator as $g$. Each party has access to a hash function for key derivation $\HKDF: \G \rightarrow \bin^{l}$ and a pseudorandom generator $\PRG : \bin^l \rightarrow (\bin \times \ZZ_q \times \ZZ_q \times \bin^{\abs{m}})$ where $\abs{m}$ is the length of the payload of an offer. Both parties are assumed to have Bitcoin wallets which can sign and validate transactions which we denote as $\GenWitness$ and $\VrfyWitness$ respectively.

\subsection{Alice's Proposal}

To make her proposal $P_0$ Alice decides the amount of coins $\bet_0$ she wants to bet on a particular event identified by $\eventid$. She provides a list $I_0$ of
$n_0$ inputs which hold at lest $\bet_0$ Bitcoin in total and an (optional) change address $\change_0$ to receive the difference. Finally, she chooses a private key $x_0 \sample \ZZ_q$ and computes her corresponding public key $X_0 \gets g^{x_0}$. She posts these on the public message board. Note that she does not yet have to decide publicly which outcome she wishes to bet on.

\begin{center}
  \fbox{
  \begin{tabular}{l|l|c}
    Symbol & Description & size (bytes) \\ \hline
    \eventid & A descriptor of the oracle and event the bet is on & \textsf{id} \\
    $I_0$ & The input that Alice is using to fund the bet & $33n$ \\
    $\bet_0$ & The amount of coins that Alice will bet & 6 \\
    $\change_0$ & A change address & 20 \\
    $X_0$ & Alice's public key & 32 \\
    & Total & 58 + 33n + \textsf{id} \\
  \end{tabular}
  }
\end{center}




\subsection{Bob's Offer}

Any user who wishes to make an offer to Alice we will name Bob. Bob chooses which outcome he wants to bet $c \in \bin$ on and how much he wants to bet on it, $bet_1$. The ratio of $\bet_0$ to $\bet_1$ determine the odds for the bet as the winner will take $\bet_0 + \bet_1$. Bob also chooses the fee $\fee$ for the bet. Note that the size of Bob's offer is $63 + 130n_1$ where $n_1$ is the number of inputs uses. With base2048 encoding we have a limit of 385 bytes per tweet so this limits Bob to two inputs.

\begin{center}
  \fbox{
  \begin{tabular}{l|l|c}
    Symbol & Description & Size (bytes) \\ \hline
    $X_1$ & Bob's public key & 32 \\ \hline
          & \emph{The following are encrypted as $e$} & \\
    $c$   &  The outcome $c \in \bin$ that Bob wants to bet on & 1 \\
    $I_1$ & The input(s) that Bob is using to fund the bet & $33n_1$  \\
    $\sigma_1$ & The witness for each input & $97n_1$ \\
    $\bet_1$ & The value that Bob will wager  & 6 \\
    $\fee_1$ & The fee that Bob includes & 4 \\
    $\change_1$ & A change address  & 20 \\
    & Total &  $63 + 130n_1$ \\
  \end{tabular}
  }
\end{center}


Bob uses the $\textsf{Offer}$ algorithm to generate his offer in response Alice's bet proposal $(X_1,e)$. First it computes the Diffie-Hellman shared secret from $X_0$ and Bob's private key $x_1$ and uses the output of $\PRG$ to encrypt the payload of the message and to randomize the public keys that appear on the blockchain including the order in which they appear. Without randomisation, it would be easy to link the public keys in the $\OPCHECKMULTISIG$ output to the keys posted on the message board. This also hides from observers which of the Bobs that Alice is actually executing the protocol with (or whether she executing it with any of them at all).

\begin{center}
  \fbox{
    \begin{pchstack}
  \procedure{\textsf{Offer}$(P_0, (I_1, \change_1,\bet_1, \fee))$}{
    x_1 \sample \ZZ_q; X_1 \gets x_1G; X \gets  X_0^{x_1} \\
    P_1 := (X_1, I_1,pk_1,\change_1,\bet_1, \fee) \\
    r \gets \HKDF(X) \\
    (b, b_0,b_1, k) \gets \PRG(r) \\
    \Fund \gets \GenFund(P_0,P_1,(b,b_0,b_1)) \\
    \sigma_1 \sample \GenWitness(I_1, \Fund) \\
    m \gets I_1 \Vert \sigma_1 \Vert \bet_1 \Vert \fee \Vert \change_1 \\
    e \gets k \oplus m \\
    \pcreturn (e, (x_1,X_1))
   }
   \pchspace
   \procedure{$\GenFund(P_0,P_1, (b,b_0,b_1))$}{
    (X_0,I_0,\change_0,\bet_0) := P_0 \\
    (X_1,I_1,\change_1,\bet_1,\fee) := P_1 \\
    Z_0 \gets X_0 + b_0G + O_{1-c} \\
    Z_1 \gets X_1 + b_1G + O_c \\
    \Fund \gets \Tx( \\
    \t \texttt{inputs:} I_0 || I_1, \\
    \t \texttt{outputs:} [ \\
       \t \t (\OPCMS(Z_b,Z_{1-b})), \bet_0 + \bet_1), \\
       \t \t (\change_0, I_0.\textsf{value} - \bet_0), \\
       \t \t (\change_1, I_1.\textsf{value} - \bet_1 - \fee) \\
       \t ]) \\
    \pcreturn \Fund
   }
  \end{pchstack}
}
\end{center}

\subsection{Alice Takes Offer}

For each offer she receives $(X_1, e)$, Alice decrypts it to check if it is deseriable to her and verifies its contents. If she wishes to accept the offer she signs her set of inputs $I_0$ with her wallet and broadcasts the $\Fund$ transaction as described in \textsf{TakeOffer} below.

\begin{center}
  \fbox{
    \procedure{\textsf{TakeOffer}$((P_0,x_0), X_1, e)$}{
      X \gets X_1^{x_0}; r \gets \HKDF(X) \\
      (b, b_0,b_1, k) \gets \PRG(r) \\
      (I_1, \sigma_1, \bet_1, \fee, \change_1) \gets k \oplus e  \\
      P_1 := (X_1, I_1, \bet_1, \fee, \change_1) \\
      \Fund \gets \GenFund(P_0,P_1,(b, b_0,b_1)) \\
      \VrfyWitness(I_1,\sigma_1, \Fund) \stackrel{?}{=} 1 \\
      \sigma_0 \sample \GenWitness(I_0, \Fund) \\
      \pcreturn ((I_0, \sigma_0), (I_1,\sigma_1), \Fund)
      }
    }
  \end{center}


\subsection{Oracle Attests to Outcome}

When the oracle decides the outcome $w \in \bin$ of the event they the discrete logarithm $o_w$ of the corresponding outcome public key $O_w$. Upon seeing $o_w$, the winner is able to compute the private key for $Z_w$ with $z_w = x_w + b_w + o_w$. With knowledge $z_w$ they construct a ECDSA signature to spend the output of the $\Fund$ transaction in a transaction of their choosing.


% \subsection{Rogue Key Attack Against the Original Scheme}

% The original DLC proposal had no formal security analysis and indeed it seems there is a rouge key attack against the scheme as originally envisioned.


% The original DLC construction required three transactions to settle. One transaction is used to lock up the funds then a ``contract execution transaction'' puts the contract into a kind of staging state -- if the CET represents the correct outcome the participants collect their corresponding funds

% Unfortunately, it seems the scheme is vulnerable to rogue key attacks. Using the notation of the orginal paper each contract execution transaction can be spent based on the following rules:

% \[ Pub_{A_i} \lor (Pub_B \lor TimeDelay) \]

% This indicates that the coins can be taken by knowing the private key for $Pub_{A_i}$ or $Pub_B$ after $TimeDelay$. $Pub_{A_i}$ is similar to our $Z_0$ above in that it is composed of a key that Alice knows and the key the oracle will release. Specifically, $Pub_{A_i} = Pub_{Alice} + s_iG$ where $s_iG$ is the scalar component of the Schnorr signature that the oracle will reveal upon outcome $i$. Thus Alice will be able to construct the private key for $Pub_{A_i}$ should the oracle reveal $s_i$.

% Unfortunately, a malicious Alice is able to take the funds without knowing $s_iG$. If Alice selects $Pub_{Alice} \gets aG - s_iG$, then she will know the discrete logarithm of $Pub_{A_i} = Pub_{Alice} + s_iG = (aG - s_iG) + s_iG = aG$. The value $s_iG$ is public information and effectively broadcasted by the oracle making this attack very easy to pull off in practice. In order for the scheme to be secure the public key $Pub_{A_i}$ needs to be constructed fairly either by Alice providing a proof of knowledge of $Pub_{Alice}$ or by through a... As we will show below, our scheme is secure under the \emph{Knowledge of Exponent} (KEA1) assumption where we assume that because both Alice and Bob are able to use  their public keys to compute a Diffie-Hellamn shared secret, they must both know the corresponding private key.



\section{Security}

We now briefly provide an informal intuition for the security and privacy properties of the protocol. First and foremost we must guarantee that only the party that wins can claim the funds. Dryja's original construction did not formally make security guarantees and unfortunetly seems to be vulnerable to \emph{rogue key} attacks commonly found in multisignature schemes. The protocol is similar to ours that that funds are locked to a combination of the user's public key and the group element representing the attestation to outcome $i$ e.g. $Z_{0,i} = X_0 + O_i$. Unfortunately, this allows the user to choose their key maliciously as $X_0 = aG - O_i$ for some $a \sample \ZZ_q$ and therefore set the discrete logarithm of $Z_{0,i}$ to $a$. There are multiple ways of fixing this issue

Our keys are computed in a similar way where for Alice, $Z_0 = X_0 + b_0G + O_{1-c}$, but because we also use $X_0$ as a Diffie-Hellman key it is difficult for her to choose their key maliciously. Specifically, under the \emph{Knowledge of Exponent} (KEA1) assumption we can extract $x_0$ from $(X_0,X)$. Note that both parties must query $\HKDF$ with $X$ which in the random oracle model allows the proof to access it. If we reasonably assume that ECDSA signatures are unforgeable, we can also assume that we can extract $z_0$ from the signature. Thus we can compute $o_{1-c} \gets z_0 - x_0 - b_0$ which contradicts our premise that Alice was able to take the coins without knowing $o_{1-c}$ (or that the discrete logarithm problem is hard).


\subsection{Privacy}

Out of necessity Alice posts her initial bet proposal on a public message board to attract offers. Observers of the blockchain and the message board will therefore be able to link the inputs Alice publicly chooses to the on-chain $\Fund$ transaction that spends them.

We cannot guarantee as much privacy for the participants as originally envisioned in the original protocol. Nevertheless there are two core privacy properties we seek to guarantee for our protocol: (i) \emph{winner privacy}, given the protocol messages, the on-chain transactions, and the outcome no adversary should be able to guess with better then  $\frac{1}{2}$ whether Alice or Bob won the bet (ii) \emph{Offer Privacy}, Given the proposal and a list $n$ offers and the on-chain transactions no adversary should be able to guess which offer was taken by Alice with better than $\frac{1}{n}$ probability.

These properties can be guaranteed under the Computational Diffie-Hellman assumption and the pseudorandomness of $\PRG$.


% Since the setup rounds are posted publicly we wish to minimize what can be learned from the observing the communication between the parties.

% \begin{itemize}
%   \item Contractual Fidelity: honest parties receive (at least) the coins they are entitled to under the rules of the contract.
%   \item Winner Privacy: No ledger observer can determine who won the bet by observing the protocol messages and the ledger.
%   \item Offer Privacy: Given the proposal, offers and blockchain transactions, all offers should be equally likely to be the chosen offer.
% \end{itemize}

%   There are two aspects of contractual fidelity in this case: (i) Alice must accept the offer and broadcast the fund transaction before she knowns the outcome, (ii) The party that claims the funds is only able to do so if the oracle released the corresponding key.


%   In order to break the fidelity of the contract the malicious party would have to gain coins without having won the bet. As we have just seen above, the original proposal does not ensure properly ensure this against malicious key generation. It is straightforward to see under the Knowledge of Exponent assumption that our scheme is secure

%   \begin{theorem}{DLOG ROM KEA1 $\implies$ Contractual Fidelity}
%     If there
%   \end{theorem}

%   \begin{proof}
%     any adversary who is able to do this is able to solve discrete logarithms. In addition we assume that there is an efficient extractor for an adversary who forges ECDSA signatures. We model $\PRG$ as a non-programmable random oracle.

% First we deal with the case of a malicious Alice, who after seeing Bob's offer, broadcasts the fund transaction and is then able to take the funds despite the Oracle not revealing $o_0$. Alice must provide a signature on $Z_0$ or $Z_1$ to spend from the $\OPCHECKMULTISIG$ 1-of-2 output. First, if she spends from $Z_1$ it is easy to see that she can solve discrete logarithms since $Z_1 = G(x_1 + b_1 + o_1)$. Since $o_1$ and $b_1$ are known to the environment it is easy to extract an arbitrary discrete logarithm of a point $Y$ simply by setting $X_1 := Y$ and returning $y = z_1 - b_1 - o_1$.

% First in order to decrypt Bob's offer she must have queried $\PRG$ with the Diffie-Hellman key $X = X_1^{x_0}$. From this we extract $x_0$ by the KEA1 assumption and also receive $b_0$ from the output of $\PRG(X)$.

% \end{proof}

% \begin{theorem}
%   An adversary $\adv$ who is able to distinguish
% \end{theorem}




% \section{The Schnorr Protocol}
% We name our two parties Alice and Bob. Alice initially proposes a event to bet on and how much she is willing to bet. After seeing her proposition people respond with how much they are willing to give Alice if she were to win (in exchange for her bet amount should they win). We call generically call each of these bidders ``Bob''. After seeing a Bid that she is satisfied with, she uses the cryptogrpahic information contained within the bid along with the private data she used to construct the following transaction scaffold:

% \begin{enumerate}
%     \item \textsf{Fund}: This spends from Alice and Bob's input into a transaction with an output locked on their joint key
%     \item \textsf{Outcome}: This spends all the coins from \textsf{Fund} to Alice's address.
%     \item \textsf{BobWin}: Spends all the coins from \textsf{Fund} to Bob's address.
% \end{enumerate}

% Alice will have a full signature on



% \subsection{Alice's Proposal}

% % Explain the motivation for each element.


% \begin{tabular}{l|l|c}
%     \textsf{event\_id} & A descriptor of the oracle and event the bet is on & - \\
%     $I^A$ & The input that Alice is using to fund the bet & 32 \\
%     $\bet^A$ & The value that Alice will wager (optional) & 8 \\
%     $\change^A$ & Her change address (optional) & 32 \\
%     $\win^A$ & The address that will receive the coins if she wins & 32 \\
%     a & Alice's choice bit for the binary outcome & 1 \\
%     $X^A$ & Alice's half of the public key for the \textsf{Fund} output & 32 \\
%     $R_0^A$ & Her signature nonce for first outcome & 32 \\
%     $R_1^A$ & Her signature nonce for the second outcome & 32 \\
%     & Total & 297 \\
% \end{tabular}

% \subsection{Bob's bid}

% Define $b := 1 - a$,i.e the other choice.

% \begin{tabular}{l|l|c}
%      $I^B$ & The input that Bob is using to fund the bet & 32  \\
%      $\textsf{bet}^B$ & The value that Bob will wager (optional) & 8 \\
%      $\textsf{change}^B$ & A change address (optional) & 32 \\
%      $\textsf{win}^B$ & The address that will receive the coins if he wins & 32 \\
%      $X^B$ & Bob's half of the public key for the \textsf{Fund} output & 32 \\
%      ($R_0^B$, $s_0^B$) & His half adaptor signature for $\textsf{Outcome}_0$ & 64 \\
%      ($R_1^B$, $s_1^B$) & His half adaptor signature for $\textsf{Outcome}_1$ & 64 \\
%      $\alpha_I^B$ & His adaptor signature on the \textsf{Fund} for his input $I^B$ (conditioned on $\alpha_{1-c}$) & 64 \\
%      & Total & 249 \\
% \end{tabular}





% \subsection{Security}
% % - Prove adaptor signatures secure: They don't help you do anything other than complete an adaptor signature, also prove that generating an adaptor signature on a particular point cannot be forged
% To cryptographic security of our protocol can be reduced to three separate claims:
% \\
% \\
% \textbf{Bob receives $\alpha_{1-c}$}: If \texttt{Fund} transaction is broadcast by Alice, then Bob can extract $\alpha_{1-c} = \alpha(X,\textsf{Outcome}_{1-c}, O_{1-c})$
% \newline
% \newline
% \textbf{Fund is only spent by Outcome}: A malicious party spends the \texttt{Fund} transaction's output with a transaction other than \texttt{Output} with negligible probability

% % - Reduce being able to spend from Fund to knowing the private key of the other party (except for Outcome transactions)
% % \textbf{Oracle decides outcome}: If a party broadcasts $\texttt{Outcome}_e$ then they know $o_e$ where $O_e = g^{o_e}$.
% % - Show how both parties can extract o_e

% %TODO:

% % - Show the Fund transaction must reveal o_e (as an aside Bob's input adaptor signature only helps her produce a signature on Fund)





\bibliography{bib.bib}{}
\bibliographystyle{splncs04}

\end{document}
